\documentclass[10pt,a4paper]{article}
\usepackage[T1]{fontenc}
\usepackage{amssymb}
\usepackage[left=2.25cm,right=2.25cm,top=2.25cm,bottom=2.75cm]{geometry}
\usepackage{graphicx}
\usepackage{isabelle}
\usepackage{isabellesym}
\usepackage[only,bigsqcap]{stmaryrd}
\usepackage{pdfsetup}

\urlstyle{tt}
\isabellestyle{it}

\renewcommand{\isacharunderscore}{\_}
\renewcommand{\isachardoublequoteopen}{``}
\renewcommand{\isachardoublequoteclose}{''}

\begin{document}

\title{Labeled Transition Systems}
\author{Anders Schlichtkrull, Morten Konggaard Schou, Ji\v{r}\'i Srba and Dmitriy Traytel}
\date{}

\maketitle

\begin{abstract}
\noindent
Labeled transition systems in computer science abound. They are used e.g. for automata and for program graphs in program analysis.
We formalize labeled transition systems with and without epsilon transitions.
In contrast to other formalizations (for an incomplete overview see \cite{graphs}) the set of nodes is formalized as a type rather than a set,
and a labeled transition system is simply represented as a locale fixing a set of transition where each transition is a triple of respectively a start note, a label and an end note.

\end{abstract}

\tableofcontents

\newpage

% sane default for proof documents
\parindent 0pt
\parskip 0.5ex

\newpage

% generated text of all theories
\input{session}

% optional bibliography
\bibliographystyle{alpha}
\bibliography{root}

\end{document}
